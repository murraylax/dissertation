\documentclass[12pt,notitlepage,oneside]{book}
\usepackage[T1]{fontenc}
\usepackage{calc}
\usepackage{setspace}
\usepackage{multicol}
\usepackage{fancyheadings}

\usepackage{chapterbib}
\usepackage{graphicx}
\usepackage{color}
\usepackage{rotating}
\usepackage{harvard}
\usepackage{aer}
\usepackage{aertt}
\usepackage{verbatim}

\setlength{\voffset}{0in}
\setlength{\topmargin}{0pt}
\setlength{\hoffset}{0in}
\setlength{\oddsidemargin}{0.5in}
\setlength{\headheight}{0pt}
\setlength{\headsep}{0in}
\setlength{\marginparsep}{0in}
\setlength{\marginparwidth}{0pt}
\setlength{\marginparpush}{0in}
\setlength{\footskip}{.2in}
\setlength{\textwidth}{6in}
\setlength{\textheight}{9in}
\setlength{\parskip}{0pc}

\renewcommand{\baselinestretch}{1.5}

\newcommand{\bi}{\begin{itemize}}
\newcommand{\ei}{\end{itemize}}
\newcommand{\be}{\begin{enumerate}}
\newcommand{\ee}{\end{enumerate}}
\newcommand{\bd}{\begin{description}}
\newcommand{\ed}{\end{description}}
\newcommand{\prbf}[1]{\textbf{#1}}
\newcommand{\prit}[1]{\textit{#1}}
\newcommand{\beq}{\begin{equation}}
\newcommand{\eeq}{\end{equation}}
\newcommand{\beqa}{\begin{eqnarray}}
\newcommand{\eeqa}{\end{eqnarray}}
\newcommand{\bdm}{\begin{displaymath}}
\newcommand{\edm}{\end{displaymath}}
\newcommand{\script}[1]{\begin{cal}#1\end{cal}}
\newcommand{\citee}[1]{\citename{#1} (\citeyear{#1})}
\newcommand{\h}[1]{\hat{#1}}
\newcommand{\ds}{\displaystyle}

\newcommand{\app}
{
\appendix
}

\pagestyle{plain}

\begin{document}

\pagenumbering{roman}

\title{THREE ESSAYS IN ADAPTIVE EXPECTATIONS IN NEW KEYNESIAN MONETARY ECONOMIES}
\author{James M. Murray}
\date{}

\maketitle
\thispagestyle{empty}
\vspace*{3.5in}
\begin{center}
\begin{singlespace}
 Submitted to the faculty of the University Graduate School

 in partial fulfillment of the requirements

 for the degree

 Doctor of Philosophy

 in the Department of Economics,

 Indiana University

 September 2008
\end{singlespace}
\end{center}
\newpage

\begin{center}
Accepted by the Graduate Faculty, Indiana University, in partial fulfillment of the requirements for the degree of Doctor of Philosophy.

\vspace{0.5in}

Doctoral Committee:

\vspace{0.5in}

\line(1,0){250}

Eric M. Leeper, Chair

\vspace{0.5in}

\line(1,0){250}

Kim P. Huynh

\vspace{0.5in}

\line(1,0){250}

Brian M. Peterson

\vspace{0.5in}

\line(1,0){250}

Todd B. Walker

\vspace{2.6in}

Date of Oral Examination:  July 8, 2008

\end{center}
\newpage

\begin{center}
\vspace*{3in}

\copyright~ 2008

James M. Murray

ALL RIGHTS RESERVED

\end{center}
\newpage

\vspace*{1in}

\noindent \textbf{Acknowledgements:} I am grateful for the advice and guidance of my dissertation committee: Eric Leeper, Kim Huynh, Brian Peterson, and Todd Walker; for useful conversations with Pedro Falc\~{a}o de Araujo, James Bullard, Troy Davig, Kenneth Kasa, Fabio Milani, Michael Plante, and Bruce Preston; and for comments by the participants of the 2007 Federal Reserve Bank of St. Louis Learning Week Conference, the 2007 Missouri Economics Conference, and Indiana University economics department seminars.  All errors are my own.
\newpage

\begin{center}
James M. Murray

\vspace*{0.5pc}

THREE ESSAYS IN ADAPTIVE EXPECTATIONS IN NEW KEYNESIAN MONETARY ECONOMIES
\end{center}

\vspace*{0.5pc}

\noindent \textbf{Abstract:} This dissertation explores the empirical significance of least squares learning in estimated New Keynesian monetary models with U.S. data.  Specifically, the papers set out to determine what impact learning has on dynamics of output, consumption, investment, inflation and monetary policy, and whether learning can explain empirical puzzles in the monetary literature such as the Great Moderation.  In the first dissertation paper, a standard New Keynesian model is estimated with constant gain learning with three specifications for how agents' expectations are initialized.  The results indicate differences in the model's prediction depending on the type of initial expectations, but the learning models do not significantly explain the data better than rational expectations.  The second paper examines an extension of the model that allows for firm-specific capital accumulation. The results show that learning can lead to very different predictions for the impacts of structural shocks, depending on the choice for agents' initial beliefs.  Again, constant gain learning is shown to not explain the Great Moderation any better than rational expectations.  The final paper examines an extension to the learning process, where the learning gain changes endogenously with agents forecast errors.  This learning framework is estimated jointly with a regime-switching volatility mechanism to determine if dynamic gain learning can lead to lower estimates for exogenously changing volatility.  The results show, rather, that learning gain dynamics are quite small and are again not capable of explaining time-varying macroeconomic volatility.

\begin{center}
\vspace*{0.15in}
\line(1,0){250}
\vspace*{0.15in}
\line(1,0){250}
\vspace*{0.15in}
\line(1,0){250}
\vspace*{0.15in}
\line(1,0){250}
\end{center}

\newpage


\tableofcontents
\listoftables
\listoffigures
\newpage

\pagenumbering{arabic}
\chapter{Initial Expectations in New Keynesian Models with Learning}

\noindent \textbf{Abstract:}
\input{../initexp/initexp_abstract.tex}
\input{../initexp/initexp_body.tex}

\newpage
\noindent\large\textbf{References}\vspace{-5pc}\normalsize
\input{../initexp/initexp_bib.tex}
\newpage

\input{../initexp/initexp_tabsfigs.tex}
\chapter{Empirical Significance of Learning in a New Keynesian Model with Firm-Specific Capital}

\noindent \textbf{Abstract:}
\input{../capital/capital1_abstract.tex}

\input{../capital/capital1_body.tex}

\newpage
\noindent\large\textbf{References}\vspace{-5pc}\normalsize
\input{../capital/capital1_bib.tex}
\newpage

\input{../capital/capital1_tabsfigs.tex}
\chapter{Regime Switching, Learning, and the Great Moderation}

\noindent \textbf{Abstract:}
\input{../badluck/badluck_abstract.tex}

\input{../badluck/badluck_body.tex}

\newpage
\noindent\large\textbf{References}\vspace{-5pc}\normalsize
\input{../badluck/badluck_bib.tex}
\newpage

\input{../badluck/badluck_tabsfigs.tex}
\appendix
\chapter{New Keynesian Model with Firm-Specific Capital: Derivations}
\input{../capital/capital1_app_diss.tex}

\newpage
\begin{singlespace}
\pagestyle{empty}
\begin{center} 
\textbf{\Large{James Murray}}\\
\textbf{Curriculum Vitae}\\
\end{center}
\small 

\begin{tabular}{p{3in} p{3in}}
\textbf{Contact Information} & \\
Department of Economics \newline
Indiana University\newline
105 Wylie Hall\newline
100 S. Woodlawn\newline 
Bloomington, IN 47405 &
Phone: (608) 738 5408\newline
E-mail: \texttt{jmmurray@indiana.edu}\newline
\textit{http://mypage.iu.edu/\~{}jmmurray/} \newline \newline
Citizenship: United States
\end{tabular} \\ \\

\begin{tabular}{p{.5in} p{.6in} p{2.5in} p{2in}}
\multicolumn{2}{l}{\textbf{Education} \newline} & \\

Ph.D. & Economics, & Indiana University & September 2008 \\
M.A. & Economics, & Indiana University & May 2004  \\
M.A. & Economics, & University of Notre Dame & May 2002 \\
B.S. & Economics, & University of Wisconsin - La Crosse & May 2000 \\
\end{tabular} \\ \\

\begin{tabular}{p{5.5in}}
\textbf{Dissertation} \newline
``Three Essays in Adaptive Expectations in New Keynesian Monetary Economies.''\newline  Advisor: Dr. Eric Leeper.
\end{tabular} \\ \\

\begin{tabular}{p{5.5in}}
\textbf{Working Papers} \\
``Initial Expectations in New Keynesians Models with Learning''\\\\
``Empirical Significance of Learning in a New Keynesian Model with Firm-Specific Capital''\\\\
``Regime Switching, Learning, and the Great Moderation'' \\\\
``Estimating the Effects of Dormitory Living on Student Performance'' with Pedro Falc\~{a}o de Araujo.
\end{tabular} \\ \\

\begin{tabular}{p{5.5in}}
\textbf{Refereed Publications} \\
``Shirking in Major League Baseball in the Era of the Reserve Clause.'' with Glenn Knowles, Michael Haupert, and Keith Sherony. \textit{Nine: A Journal of Baseball History and Social Policy Perspectives.}  Volume 9. Spring 2001. \\
\end{tabular} \\ \\

\begin{tabular}{p{5.5in}}
\textbf{Non-Refereed Publications} \\
``Expectations for Monetary Policy.'' \textit{Business Connection.}  April 2008. \\\\
``Economic Outlook for Bio-Fuels.'' \textit{Business Connection.}  February 2008. \\
\end{tabular} \\ \\

\begin{tabular}{p{6in}}
\textbf{Research Interests}\newline 
Learning and Expectations \newline
Applied Macroeconometrics \newline
Monetary Economics \newline
Scholarship of Teaching and Learning\newline
\end{tabular} \\ 

\begin{tabular}{p{5.5in}}
\textbf{Conference and Seminar Presentations} \\
Learning Week Conference, St. Louis Federal Reserve Bank, July 2007.\newline
\indent ``Empirical Significance of Learning in a New Keynesian Model with Firm-Specific Capital'' \\\\
Indiana University Economics Department Brown Bag Workshop, May 2007.\newline
\indent ``Empirical Significance of Learning in a New Keynesian Model with Firm-Specific Capital''\\\\
Jordan River Conference, Indiana University, April 2007.\newline
\indent ``Empirical Significance of Learning in a New Keynesian Model with Firm-Specific Capital''\\\\
Jordan River Conference, Indiana University, April 2007.\newline
\indent Discussion of Allaby, ``Feasibility of Corn Ethanol from a Land Use Perspective.''\\\\
Missouri Economics Conference, University of Missouri, March 2007.\newline
\indent ``Empirical Significance of Learning in a New Keynesian Model with Firm-Specific Capital''\\\\
Indiana Academy of Social Sciences Annual Meeting, October 2006.\newline
\indent ``Empirical Significance of Learning and the Consequences of Mis-specifying Expectations''\\\\
Jordan River Conference, Indiana University, April 2006.\newline
\indent ``Empirical Significance of Learning and the Consequences of Mis-specifying Expectations''\\\\
Jordan River Conference, Indiana University, April 2005.\newline
\indent ``Liquidity in a Two Country Open Economy Model: Evidence from United States and Germany''\\\\
\end{tabular} 

\begin{tabular}{p{5.5in}}
\textbf{Awards} \\
Jordan River Conference Best Graduate Student Paper Award, April 2007. \\\\
Future Faculty Teaching Fellowship, 2007. \\
\end{tabular} \\ \\

\newpage

\begin{tabular}{p{6in}}
\textbf{Teaching Interests} \newline
\textit{Undergraduate:}\newline
Principles and Intermediate Macroeconomics \newline
Principles and Intermediate Microeconomics \newline
Elementary Statistics\newline
Statistical Methods / Econometrics \newline
Monetary Economics \newline
International Economics / Finance \newline 
Mathematical Economics \newline 
Operations Research \newline \newline
\textit{Graduate:}\newline
Statistical Methods / Research Methods \newline
Applied Econometrics \newline
Macroeconomics \newline
Monetary Economics \newline
Open Economy Macroeconomics \newline
Computational Economics \newline
Operations Research \newline
\end{tabular} \\

\begin{tabular}{p{2.5in} p{1.7in} p{1.5in}}
\textbf{Employment} \\
Teaching Fellow & IUPU - Columbus & 8/2007 - 5/2008 \\
Adjunct Professor & Viterbo University & 5/2007 - 8/2007 \\
 & & 5/2006 - 8/2006 \\
Associate Instructor & Indiana University & 9/2003 - 5/2007 \\
Adjunct Professor & Ivy Tech State College & 3/2004 - 8/2004 \\
Teaching and Research Assistant & Indiana University & 9/2002 - 5/2003 \\
Teaching and Research Assistant & University of Notre Dame & 9/2000 - 8/2002 \\
Intern Computer Programmer & Trane Company & 8/1999 - 8/2000 \\
\end{tabular} \\ \\

\begin{tabular}{p{3in}p{3in}}
\textbf{Teaching Experience} & \\
\emph{Primary Instructor:} & \\
Econ 202: Principles of Macroeconomics & IUPU - Columbus, 1 Semester \\
Econ 270: Introductory Statistics & IUPU - Columbus, 2 Semesters \\
Math 130: Introductory Statistics & Viterbo University, 2 Sessions \\
Econ E201: Principles of Microeconomics & Indiana University, 6 Semesters \\
Econ E202: Principles of Macroeconomics & Indiana University, 2 Semesters \\
Econ E322: Intermediate Macroeconomics & Indiana University, 2 Summers \\
Econ E201: Principles of Macroeconomics & Ivy Tech State College, 1 Session \\
Econ E202: Principles of Microeconomics & Ivy Tech State College, 1 Session \\ \\
\emph{Teaching Assistant:} & \\
Econ E472: Econometrics II & Indiana University, 1 Semester \\
Econ S370: Honors Statistics & Indiana University, 1 Semester \\
Econ E471: Econometrics I & Indiana University, 1 Semester \\
Econ 201: Principles of Microeconomics & University of Notre Dame, 1 Semester \\
Econ 592: Graduate Econometrics I & University of Notre Dame, 1 Semester \\
\end{tabular} 
\newpage

\thispagestyle{empty}
\begin{tabular}{p{5.5in}}
\textbf{Professional Development} \newline
AIMS (Adapting Innovative Materials for Statistics) Workshop, Minneapolis, MN.  July 2008.\\\\
Service Learning Workshop hosted by Indiana University Purdue University - Indianapolis Center for Service and Learning.  January 2008.\\\\
Indiana University FACET (Faculty Colloquium on Excellence in Teaching) Summer Institute.\newline July 2007.
\end{tabular} \\ \\

\begin{tabular}{p{5.5in}}
\textbf{Professional/Academic Service} \newline
Economics Candidate Search and Screen Committee.  Indiana University Purdue University - Columbus. Fall 2007 - Spring 2008.
\end{tabular} \\ \\

\begin{tabular}{p{5.5in}}
\textbf{Community Service} \\
Mentor for Big Brothers Big Sisters of Columbus, IN.  January 2008 - May 2008.\\\\
Bartholomew Consolidated School Corporation (BCSC) Book Buddy.  September 2007 - April 2008.\\\\
Science/Inventor Fair judge.  Central Middle School, Columbus, IN.  February 2008. 
\end{tabular} \\ \\

\begin{tabular}{p{2in} p{3.3in}}
\textbf{Computer Skills} & \\
Teaching Tools: & Blackboard, Oncourse CL, MyEconLab, CourseCompass, Discover Econ, E-instruction CPS (Classroom Performance System), Interwrite PRS (Personal Response System). \\
Econometrics / Statistics: & MatLab, Gauss, Octave, Maple, Stata, EViews, Limdep, RATS, SAS, SPSS, Statistica, GNU Scientific Libraries for C, LAPACK (Linear Algebra Package for Fortran). \\
Programming Languages: & C, C++, MS Visual C++, MS Visual Basic, Java, Fortran, Eiffel, Perl. \\
Web Programming: & Java, Perl, HTML, Javascript. \\
Operating Systems: & Linux, Unix, Windows.\\
Other: & \LaTeX, Beamer for \LaTeX, Emacs, SQL, Unix shell programming (BASH, CSH), MS Office. \\
\end{tabular} 
\newpage

\thispagestyle{empty}
\begin{tabular}{p{2.5in}p{2.5in}}
\textbf{References} & \\
\emph{Teaching References} &  \\
Dr. Jan Eriksen \newline
Dean \newline
School of Adult Learning \newline
Viterbo University \newline
900 Viterbo Drive \newline
La Crosse, WI 54601 \newline
Phone: (608) 796-3087\newline
E-mail: \texttt{jperiksen@viterbo.edu} &

Dr. James Self\newline
Lecturer of Economics \newline
Department of Economics \newline
Indiana University \newline
105 Wylie Hall\newline
100 S. Woodlawn\newline
Bloomington, IN 47405\newline
Phone: (812) 855-7590\newline
E-mail: \texttt{jkself@indiana.edu} \\ \\ \\

\emph{Research References} &  \\
Dr. Eric Leeper \newline
Professor of Economics \newline
Department of Economics \newline
Indiana University \newline
105 Wylie Hall\newline
100 S. Woodlawn\newline
Bloomington, IN 47405\newline
Phone: (812) 855-9157\newline
E-mail: \texttt{eleeper@indiana.edu} &

Dr. Kim Huynh\newline
Assistant Professor of Economics \newline
Department of Economics \newline
Indiana University \newline
105 Wylie Hall\newline
100 S. Woodlawn\newline
Bloomington, IN 47405\newline
Phone: (812) 855-2288\newline
E-mail: \texttt{kphuynh@indiana.edu} \\
\end{tabular}
\end{singlespace}


\end{document}



